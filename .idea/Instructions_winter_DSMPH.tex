
%\title{Using an image as a document header}
% From http://tex.stackexchange.com/questions/34040/graphics-logo-in-headers
\documentclass[13pt,a4paper]{article}
\usepackage[english,main=bulgarian]{babel}
\usepackage{hyperref}
\usepackage{fancyhdr}
\usepackage{awesomebox}

\begin{document}
    \title{ Информация за провеждане на зимната изпитна сесия
    }
    \date{2021/2022г.}
    \author{ Катедра "Социална медицина и обществено здраве"}
    \maketitle


    Във връзка с епидемичната обстановка в страната и решение на КС №31/14.12.2021г. , изпитите към катедра „Социална медицина и обществено здраве“ ще бъдат проведени дистанционно под формата на тест чрез използването на утвърдената от отдел ИКО платформа \emph{Microsoft Teams} за осъществяване на видео и аудио контрол на студентите по време на изпита.

    \thispagestyle{fancy}

    \section*{Технически изисквания и указания за участие на студентите в изпита}
    \begin{enumerate}

        \item Компютър/таблет/смарт телефон с включена камера и микрофон през цялото време на изпита
        \item Интернет връзката със скорост на даунлоуд и ъплоуд не по-ниска от 15 Mbps. Можете да проверите скоростта си тук:\href{https://www.speedtest.net/}{www.speedtest.net}
        \item Всеки студент е предварително разпределен в група до 35 човека за видео и аудио контрол. Списък с разпределението на студентите и точен час на явяване ще бъде публикуван в Microsoft Teams от асистента, преподавал изпитната дисциплина, най-късно до 1 ден преди съответният изпит и на \href{https://muplovdivbg.sharepoint.com/teams/FacultyofPublicHealth/Shared%20Documents/Forms/AllItems.aspx?csf=1&web=1&e=7TlGGB&cid=c99189b7%2Dd836%2D47cc%2Da5a2%2D3d04885bfeb1&RootFolder=%2Fteams%2FFacultyofPublicHealth%2FShared%20Documents%2F%D0%A1%D0%BE%D1%86%D0%B8%D0%B0%D0%BB%D0%BD%D0%B0%20%D0%BC%D0%B5%D0%B4%D0%B8%D1%86%D0%B8%D0%BD%D0%B0%20%D0%B8%20%D0%BE%D0%B1%D1%89%D0%B5%D1%81%D1%82%D0%B2%D0%B5%D0%BD%D0%BE%20%D0%B7%D0%B4%D1%80%D0%B0%D0%B2%D0%B5%20%2D%20Social%20medicine%20and%20public%20health&FolderCTID=0x012000DAF9907FD71CC142959689FC483FA4AC}{следния линк}
        \item 	Посоченият час за началото на изпита е в българско време. Студентите трябва да са онлайн в Microsoft Teams 15 минути преди началото на изпита
        \item\textbf{ Преди началото на изпита}, студентите трябва да са отворили в браузър сайта \href{https://www.office.com/}{https://www.office.com/} и да са се вписали успешно в него със своя университетски мейл акаунт и парола. Ползването на акаунт, различен от този, предоставен от МУ-Пловдив е основание за анулиране на изпита
        \item 	\textbf{В началото на изпита}, студентите трябва да са включили своите уеб камери и микрофони, за да се провери присъствието поименно от съответния квестор и да се запознаят с правилата за протичане на изпита;
        \item  Не се позволяват разговори и коментари по-време на попълване на тест
        \item  Всеки студент има достъп до пробна изпитна форма за апробация и запознаване с типа въпроси и работа със системата на \href{https://forms.office.com/Pages/ResponsePage.aspx?id=6myJOQCkhkerbLuRZH_g-R6xkpuTW9FAl9MVrsryF45UQTFKQkU4UjVQOFYzTkVUVVFVVzhGWVRBUy4u&wdLOR=c23E40142-2058-4E02-807D-BF3DE33599A8}{следния линк}
        \item Времето за теста започва от момента на споделяне на линка в общия чат на групата. Квесторите напомнят оставащото време на всеки 10 минути.
        \item Резултатите се обявяват до 1 седмица след изпита в MS Teams
    \end{enumerate}
    Тестът ще бъде проведен чрез използването на Microsoft Forms от Office 365. Изпитните тестове се състоят от 50 въпроса, носещи общо 100 точки.
    \section*{Критерии за анулуране на тест}
    \warningbox{При подаване на тест след регламентираното време за попълване}
    \warningbox{При използване на мейл акаунт, различен от личния Ви университетски}
    \warningbox{При неразрешени разговори и коментари след началото на теста}
    \warningbox{При отказ на студента да включи микрофон и камера, за да бъде идентифициран}
    \warningbox{При разпадане на връзката на студента повече от 3 мин. по-време на започнал вече тест}
    \warningbox{При установено използване на помощни материали по-време на провеждане на тест (различни от формули и калкулатор за изпит по статистика)}
    \warningbox{При използване на булет знаци в текстовите отговори}
    \section*{При разпадане на онлайн връзката}
    \tipbox{Ако се разпадне връзката на екзаминатора, студентите следва да запазат спокойствие и да изчакат екзаминатора в MS Teams без да напускат или спират микрофони и камера}
    \tipbox{Ако се разпадне връзката на студент преди началото на теста, студентът може да се включи и да продължи изпита си}
    \importantbox{При разпадане на връзката на студент до 3 мин. по-време на попълване на тест е възможно да продължи с изпита без да му бъде анулиран}

    \section*{Правила преди и по време на изпита}
    \begin{itemize}
        \item Подготви данните си (мейл и парола) за вход в Office 365
        \item Отвори в браузър сайта \href{https://www.office.com/}{https://www.office.com/} и се впиши успешно в него със своя университетски мейл акаунт и парола
        \item Увери се, че устройствата ти (компютър, таблет или телефон) са със заредени батерии или включени в мрежата
        \item Провери интернет скоростта си \href{https://www.speedtest.net/}{www.speedtest.net}
        \item Провери дали браузърът или приложението са последната актуална версия. Ако тестът не се отваря, въпреки описаните стъпки, опитайте да го отвроте с алтернативен браузър (firefox, explorer)
        \item Деактивирай AVAST антивирусна програма (ако използваш такава)
        \item Бъди готов и влез в MS Teams поне 30 минути преди обявения час за изпит
        \item Провери камерата и микрофона си дали работят (MS Teams)
        \item Присъедини се към срещата поне 15 минути преди началото на изпита
        \item Изслушвай внимателно указанията от твоя екзаминатор
        \item След получаване на линка за теста в чата на срещата в MS Teams, в браузъра трябва да бъде отворена само уебстраницата към изпитния тест
        \item Свържи се с твоя екзаминатор, ако имаш затруднения като вдигнеш ръка (MS Teams)
        \item Не говори и не прави излишни коментари по-време на попълване на тест
        \item Екзаминаторът ще напомня за оставащото време
        \item Разпредели времето си така, че да може да провериш накрая отговорите си за грешки
        \item Когато си готов, предай теста си с натискане на бутона \emph{„подай“}
        \item След приключване на теста, не излизай от срещата, защото екзаминаторът ще изпрати анкетна карта, която трябва да се попълни задължително!
        \section*{Информация за изпити}
        \subsection*{Социална медицина и медицинска етика I част, специалност медицина}
        \awesomebox{0pt}{\faCogs}{black}{Изпитът ще се проведе на \emph{24.януари.2022г.от 10 ч.}. Тестът се състои от:
            \begin{itemize}
                \item 	\textbf{20 затворени въпроса} по \emph{медицинска етика} със само един верен отговор (всеки верен отговор носи 1 т.);
                \item \textbf{5 отворени въпроса} по \emph{медицинска етика}, изискващи кратък текстови отговор (6 т. при пълен и изчерпателен отговор);
                \item  \textbf{20 затворени въпроса} по \emph{медицинска статистика} с един верен отговор само (всеки верен отговор носи 1 т.);
                \item  \textbf{5 отворени въпроса} по \emph{медицинска статистика}, изискващи кратък текстови отговор (6 т. при пълен и изчерпателен отговор).
            \end{itemize}
            Времето за попълване на теста е 50 минути.}
    \end{itemize}

    \notebox{ Оценяване
        \begin{itemize}
            \item 0 - 60 точки: Слаб (2)
            \item 61 - 70 точки: Среден (3)
            \item 71 - 80 точки: Добър (4)
            \item 81 - 90 точки: Много добър (5)
            \item 91 - 100 точки: Отличен (6)
        \end{itemize}
    }



    \awesomebox[violet]{2pt}{\faRocket}{violet}{ Важни линкове и материали
        \begin{itemize}
            \item \href{https://drive.google.com/file/d/19ecAdsGAGK3eIVpIEnpwE-a9-de1usVL/view?usp=sharing}{Конспект}
            \item \href{https://muplovdivbg.sharepoint.com/:f:/t/I158/EsbVVmg_YfBIgQT3apbh6K8Bg285UNJioANFodNHljw5DA?e=l0BCOZ}{Лекции \emph{Медицинска статистика}}
            \item \href{https://muplovdivbg.sharepoint.com/:f:/t/I158/Ega_fv4AYepOjiqSdq9mqc4BqpYNe447CvwREXCnxLbQBQ?e=zYyMES}{Лекции \emph{Медицинска етика}}
            \item \href{https://muplovdivbg.sharepoint.com/:f:/t/I158/EjaKEUa5P8pPgaRgoHIYD18BA8m5VpmWJkZ9eb5MdMhX0A?e=mwpcre}{Упражнения \emph{доц.Искров, дм}}
            \item \href{https://muplovdivbg.sharepoint.com/:f:/t/I158/Eg1Od1V_0IpJrc3aCkla-QYBpP_k0j8mDSpEE_lxqmf42g?e=IBtKbv}{Упражнения \emph{гл.ас. Р.Райчева, дм}}
            \item \href{https://muplovdivbg.sharepoint.com/:f:/t/I158/EoDeWj3wupNEt3EoChGG_L0BjJVyFOxjd06vJNNzwCIcqA?e=ZxbtMc}{Упражнения \emph{ас.д-р Костадинов}}
            \item \href{https://muplovdivbg.sharepoint.com/:b:/t/I158/EbSvEJXYWLVCjFoIwACOVs0BU7zx-Gr6vnf5bmmK-5TM1Q?e=ZB5Bfv}{Формули по \emph{Медицинска статистика}}
            \item \href{https://muplovdivbg.sharepoint.com/:b:/t/I158/EYhg0gEIKz1KuLP_MDf0hUEBtaE74Kg_HyuUGg8opWv9WA?e=4VQrQj}{Резултати от текущ контрол}

        \end{itemize}
    }

    \subsection*{Социална медицина и медицинска етика, специалност дентална медицина}

    \awesomebox{0pt}{\faCogs}{black}{Изпитът ще се проведе на \emph{06.януари.2022г.от 10 ч.}. Тестът се състои от:
        \begin{itemize}
            \item 	\textbf{20 затворени въпроса} по \emph{социална медицина} със само един верен отговор (всеки верен отговор носи 1 т.);
            \item \textbf{5 отворени въпроса} по \emph{социална медицина}, изискващи кратък текстови отговор (6 т. при пълен и изчерпателен отговор);
            \item  \textbf{20 затворени въпроса} по \emph{медицинска статистика} с един верен отговор само (всеки верен отговор носи 1 т.);
            \item  \textbf{5 отворени въпроса} по \emph{медицинска статистика}, изискващи кратък текстови отговор (6 т. при пълен и изчерпателен отговор).
        \end{itemize}
        Времето за попълване на теста е 50 минути.}

    \notebox{ Оценяване
        \begin{itemize}
            \item 0 - 60 точки: Слаб (2)
            \item 61 - 70 точки: Среден (3)
            \item 71 - 80 точки: Добър (4)
            \item 81 - 90 точки: Много добър (5)
            \item 91 - 100 точки: Отличен (6)
        \end{itemize}
    }



    \awesomebox[violet]{2pt}{\faRocket}{violet}{ Важни линкове и материали
        \begin{itemize}
            \item \href{https://drive.google.com/file/d/1M-V3Kl_cbQ9qWVMnomMARbOZOjyHQpL0/view?usp=sharing}{Конспект}
            \item \href{https://muplovdivbg.sharepoint.com/:f:/t/msteams_77775d/EuSKnKSBUYBPitRZIVYKJOgBMMP-CHfBguviq0J2pBPbuQ?e=xWCB0e}{Лекции \emph{Медицинска статистика}}
            \item \href{https://muplovdivbg.sharepoint.com/:f:/t/msteams_77775d/EpDo-4be829Fk-bAomj3H98BqSYMPGqKL4awI2j98LfVOg?e=qo3Jcw}{Лекции \emph{Социална медицина}}
            \item \href{https://muplovdivbg.sharepoint.com/:f:/t/msteams_77775d/EkhllIATaoVNpIBASIjeQAgBCVb0tcW88E4Nlr0FoJWmVw?e=2pXU8c}{Упражнения \emph{Медицинска статистика}}
            \item \href{https://muplovdivbg.sharepoint.com/:f:/t/msteams_77775d/EisRr4kMgyJOtAoRSh6SlfkBWpluHZUTq19QkCtfVXyGUQ?e=WgoysZ}{Упражнения \emph{Социална медицина}}
            \item \href{https://muplovdivbg.sharepoint.com/:b:/t/msteams_77775d/EZ5_JqTGv8BLjay0QoJT4MIBc2x3f1arlS-Dsaup2bHrbQ?e=pkvGGM}{Формули по \emph{Медицинска статистика}}
            \item \href{https://muplovdivbg.sharepoint.com/:b:/t/msteams_77775d/EXqxOHjhbJZNmTNDWwX1rTABug_Fb5ul509M8xQc2sqbiQ?e=BblzSa}{Класирани реферати}

        \end{itemize}
    }

\end{document}